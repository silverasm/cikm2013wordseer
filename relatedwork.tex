




\section{Related Work}

Exploratory data analysis  (EDA) is an approach popularized by the statistician John Tukey whose goals are to maximize insight into a data set, uncover underlying structure, test underlying assumptions, and find bases from which to formulate hypotheses to test with confirmatory approaches \cite{tukey1977exploratory,tukey1980we}.  Many of the tools developed for EDA such as Polaris (which because the product Tableau) \cite{stolte2002polaris}, and Spotfire \cite{ahlberg1996spotfire} allow users to view multidimensional data from different angles by selecting subsets of data, viewing those as visualizations, moving laterally to view other subsets of data, moving into another view, expanding the viewed data by relaxing constraints, and so on.  However, these tools operate over numerical and categorical data, but do not seamlessly operate over raw textual information with the same flexibility. 


 Early tools such as Protofoil \cite{rao1994protofoil} made first steps into linking text search, category search, grouping, and clustering.  More recent tools such as Jigsaw \cite{stasko2008jigsaw},  Takmi (made into the product IBM TextMiner) \cite{uramoto2004text}, and SAS Text Analytics \cite{} integrate analysis techniques like text classification, named-entity recognition, sentiment analysis and summarization into one interface. 
 To make the results of these computational analyses interpretable, they display the outputs with interactive data visualization. These systems have made advanced text mining and visualization algorithms available to users without expertise in those areas, but do not provide flexibility of  access to the text of the system described here.   The TRIST search and sensemaking system \cite{jonker2005information} provides a compelling interfaces to help analysts selection a subset of documents from a large collection for further scrutiny, including extracting important entities and organizing retrieved documents into topics, but with less emphasis on analyzing text at the word level.

There are a number of exploratory tools for bibliographic citations, including Paperlens \cite{lee2005understanding} and Apolo \cite{chau2011apolo}, two  of many tools for exploring collections of bibliographic citations.  These provide access to network structure of authors, and relations among metadata categories, but do not provide rich access to the text itself.

In the digital humanities space, the Subjunctive interface \cite{bron2012subjunctive} is designed for Media Studies researchers and allows for two side-by-side comparisons of text content, but only with a limited set of views (bar charts and line graphs of frequencies along with word clouds).  WordHoard \cite{} 
Featurelens \cite{don2007discovering}
TAPor \cite{rockwell2003text}