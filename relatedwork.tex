




\section{Related Work}

Exploratory data analysis  (EDA) is an approach popularized by the statistician John Tukey whose goals are to maximize insight into a data set, uncover underlying structure, test underlying assumptions, and find bases from which to formulate hypotheses to test with confirmatory approaches \cite{turkey1977exploratory,tukey1980we}.  Many of the tools developed for EDA such as Polaris (which because the product Tableau) \cite{stotle2002polaris}, and Spotfire \cite{ahlberg1996spotfire} allow users to view multidimensional data from different angles by selecting subsets of data, viewing those as visualizations, moving laterally to view other subsets of data, moving into another view, expanding the viewed data by relaxing constraints, and so on.  However, these tools operate over numerical and categorical data, but do not seamlessly operate over raw textual information with the same flexibility. 


 Early tools such as Protofoil \cite{rao1994protofoil} made first steps into linking text search, category search, grouping, and clustering.  More recent tools such as Jigsaw \cite{stasko2008jigsaw},  Takmi (made into the product IBM TextMiner) \cite{uramoto2004text}, and SAS Text Analytics \cite{} integrate analysis techniques like text classification, named-entity recognition, sentiment analysis and summarization into one interface. 
 To make the results of these computational analyses interpretable, they display the outputs with interactive data visualization. These systems have made advanced text mining and visualization algorithms available to users without expertise in those areas, but do not provide flexibility of  access to the text of the system described here.   The TRIST search and sensemaking system \cite{jonker2005information} provides a compelling interfaces to help analysts selection a subset of documents from a large collection for further scrutiny, including extracting important entities and organizing retrieved documents into topics, but with less emphasis on analyzing text at the word level.

There are a number of exploratory tools for bibliographic citations, including Paperlens \cite{lee2005understanding} and Apolo \cite{chau2011apolo}, two  of many tools for exploring collections of bibliographic citations.  These provide access to network structure of authors, and relations among metadata categories, but do not provide rich access to the text itself.

In the digital humanities space, the Subjunctive interface \cite{bron2012subjunctive} is designed for Media Studies researchers and allows for two side-by-side comparisons of text content, but only with a limited set of views (bar charts and line graphs of frequencies along with word clouds).  WordHoard \cite{} 
Featurelens \cite{don2007discovering}
Tapoir \cite{rockwell2003text}



@article{rockwell2003text,
  title={What is text analysis, really?},
  author={Rockwell, Geoffrey},
  journal={Literary and linguistic computing},
  volume={18},
  number={2},
  pages={209--219},
  year={2003},
  publisher={ALLC}
}

@inproceedings{don2007discovering,
  title={Discovering interesting usage patterns in text collections: integrating text mining with visualization},
  author={Don, Anthony and Zheleva, Elena and Gregory, Machon and Tarkan, Sureyya and Auvil, Loretta and Clement, Tanya and Shneiderman, Ben and Plaisant, Catherine},
  booktitle={Proceedings of the sixteenth ACM conference on Conference on information and knowledge management},
  pages={213--222},
  year={2007},
  organization={ACM}
}

@inproceedings{bron2012subjunctive,
  title={A subjunctive exploratory search interface to support media studies researchers},
  author={Bron, Marc and van Gorp, Jasmijn and Nack, Frank and de Rijke, Maarten and Vishneuski, Andrei and de Leeuw, Sonja},
  booktitle={Proceedings of the 35th international ACM SIGIR conference on Research and development in information retrieval},
  pages={425--434},
  year={2012},
  organization={ACM}
}
@inproceedings{jonker2005information,
  title={Information triage with TRIST},
  author={Jonker, David and Wright, William and Schroh, David and Proulx, Pascale and Cort, Brian},
  booktitle={2005 International Conference on Intelligence Analysis},
  pages={2--4},
  year={2005}
}

@inproceedings{rao1994protofoil,
  title={Protofoil: storing and finding the information worker's paper documents in an electronic file cabinet},
  author={Rao, Ramana and Card, Stuart K and Johnson, Walter and Klotz, Leigh and Trigg, Randall H},
  booktitle={Proceedings of the SIGCHI conference on Human factors in computing systems: celebrating interdependence},
  pages={180--185},
  year={1994},
  organization={ACM}
}

@inproceedings{lee2005understanding,
  title={Understanding research trends in conferences using PaperLens},
  author={Lee, Bongshin and Czerwinski, Mary and Robertson, George and Bederson, Benjamin B},
  booktitle={CHI'05 extended abstracts on Human factors in computing systems},
  pages={1969--1972},
  year={2005},
  organization={ACM}
}

@inproceedings{chau2011apolo,
  title={Apolo: making sense of large network data by combining rich user interaction and machine learning},
  author={Chau, Duen Horng and Kittur, Aniket and Hong, Jason I and Faloutsos, Christos},
  booktitle={Proceedings of the 2011 annual conference on Human factors in computing systems},
  pages={167--176},
  year={2011},
  organization={ACM}
}

@article{uramoto2004text,
  title={A text-mining system for knowledge discovery from biomedical documents},
  author={Uramoto, Naohiko and Matsuzawa, Hirofumi and Nagano, Tohru and Murakami, Akiko and Takeuchi, Hironori and Takeda, Kohichi},
  journal={IBM Systems Journal},
  volume={43},
  number={3},
  pages={516--533},
  year={2004},
  publisher={IBM}
}

@techreport{ibmcontent,
  title={IBM Content Analytics Version 2.2: Discovering Actionable Insight from Your Content},
  author={Zhu, W.-D. and Iwai, A. and Layba, T. and Magdalen, J. and McNeil, K. and Nasukawa T. and Patel, N. and Sugano, K.},
  booktitle={IBM Redbooks},
  year={2011},
  organization={IBM}
}

@article{stasko2008jigsaw,
  title={Jigsaw: supporting investigative analysis through interactive visualization},
  author={Stasko, John and G{\"o}rg, Carsten and Liu, Zhicheng},
  journal={Information visualization},
  volume={7},
  number={2},
  pages={118--132},
  year={2008},
  publisher={SAGE Publications}
}

@inproceedings{donn1996query,
  title={Query previews in networked information systems},
  author={Donn, K and Plaisant, Catherine and Shneiderman, Ben},
  booktitle={Research and Technology Advances in Digital Libraries, 1996. ADL'96., Proceedings of the Third Forum on},
  pages={120--129},
  year={1996},
  organization={IEEE}
}

@article{ahlberg1996spotfire,
  title={Spotfire: an information exploration environment},
  author={Ahlberg, Christopher},
  journal={ACM SIGMOD Record},
  volume={25},
  number={4},
  pages={25--29},
  year={1996},
  publisher={ACM}
}

@article{stolte2002polaris,
  title={Polaris: A system for query, analysis, and visualization of multidimensional relational databases},
  author={Stolte, Chris and Tang, Diane and Hanrahan, Pat},
  journal={Visualization and Computer Graphics, IEEE Transactions on},
  volume={8},
  number={1},
  pages={52--65},
  year={2002},
  publisher={IEEE}
}

@article{tukey1980we,
  title={We need both exploratory and confirmatory},
  author={Tukey, John W},
  journal={The American Statistician},
  volume={34},
  number={1},
  pages={23--25},
  year={1980},
  publisher={Taylor \& Francis}
}

@article{tukey1977exploratory,
  title={Exploratory data analysis},
  author={Tukey, John W},
  journal={Reading, MA},
  volume={231},
  year={1977}
}