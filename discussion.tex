




\section{Discussion and Future Work}

These case studies have illustrated the use of the text manipulation and visualization functionality of WordSeer to help scholars make new discoveries about the world and create new understanding from text collections; these are discoveries that most likely they would not have been able to make with any other existing tool.

Prior to using WordSeer, C.F. had general ideas about the points he wanted to make based on years of reading evidence in the field, but no way to quantify or codify those ideas based on data.  A few sessions using the tool allowed him to find changes in use in language that reflected historians' understanding of the changing attitudes between countries' relationships (China's transformation from a `card' played by American diplomats during the cold war,  to its present-day incarnation as a `central' and `entangled' partner of the United States), but with more precision (the steady increase in growth-related verbs over the last 30 years, the sudden jump in  phrases like ``US and China'' after 1994) and with more nuance (the emergence of words relating to the `centrality' and `interdependence'  of their relationship in the 2000s).


Prior to using WordSeer, R.G. had spent hours reading and thinking about the literacy essays, but had not looked at them in terms of word frequency or syntactic structure.  Using WordSeer, he was able to see the documents in a new light and form new  understanding of the general trends in his students' process of learning to write.   He was also able to formulate and find evidence for a hypothesis about an increase in proficiency with advanced writing structures as they moved to successively advanced educational levels.   He used a complex combination of searching for grammatical structures, side-by-side comparisons of sets of sentences according to adjectives and their use in context, selecting a subset of words and comparing their frequency across a metadata classification.  

A key component of the use of the tool in these examples is the freedom it affords the user to pivot on words, on words' relations to other words, to create groups of words and cross them with arbitrary metadata categories, and to be able to immediately view  the context of their original sentences and documents, thus allowing ``close reading" as part of the text analysis process.

\subsection{Areas for Improvement}
As an exploratory data analysis (EDA) system, WordSeer aids in the formulation of hypotheses, and the accumulation of evidence in favor or in refutation of hypothesis. However, it is important when doing EDA to externally verify any hypotheses formed, preferably with evidence external to the text.   A significant improvement to the tool would be a way to help users try to disprove any hypothesis that they think the tool has helped them to find. For instance, Grimmer \cite{grimmer} describes methods to validate hypotheses formed from politician's press releases, such as comparing categories formed from these documents to candidate's voting records.  Assessments of literacy essays can be compared to learning outcomes and grades, for example.  

It is also important to note that analysts must keep themselves honest, and look for evidence to counteract their claims. C.F., for example, must be sure to seek examples of word usage that go against his views on US-China relations. 

There are several improvements that need to be made to the user interface. The first problem is that our window-management panel is rigid in structure, opening up views side by side from left to right. This makes having more than two or three simultaneous panels impractical on most displays. A more adaptive, customizable window management system would make transitioning to a new views and managing old views more frictionless.

Second, the user's history of actions should be made more accessible. At present, previous views can only be revisited can via back and forward buttons on each panel, and by opening them up individually from the list of past actions. The system should instead give the user a sense of the different sensemaking paths they have taken, and allow them to replay their history, and branch off from it.

The third problem is that users cannot customize their views according to their needs. The panels always display the same three overviews: the metadata on the left, and the most frequent phrases, nouns, verbs, and adjectives on the bottom. The user should be able to choose to expand or collapse these overviews by default, as well as the details of the overviews themselves: perhaps they would like to see the most most distinctive words in the slice, instead of the most frequent. The overviews also need to make it clear that the the counts they show are the number of matching \emph{sentences}, and not the number of matching \emph{documents}, and to give users a way of switching between the two. 
 
 \subsection{Requested Features}
 Both C.F. and R.G. expressed the desire to search for sentence structures beyond word-to-word relationships. C.F. was interested in sentences with a comparative structure, specifically comparisons between  Chinese and Japanese things. As a composition instructor, R.G. was interested in the broader applications of a ``sentences structure search'' to his teaching.  If there was a way to summarize the most frequent types of sentence-structure mistakes students made, or to find overall commonalities in the ways students structured their sentences, he could focus on helping the students improve in those areas.
 
Both scholars also wanted a way to automatically create groups based on a set of examples, and to fine-tune the results by giving feedback. For example, C.F. could have used it to identify a more comprehensive set of sentences describing the complexities of the US-China relationship, and R.G. could have used it to automatically build up groups of the different senses of the word ``get'': the sense of becoming (`I got ready to \ldots'), and the sense of acquisition (`I got a lot of practice \ldots'). This would have made it much easier to see how often these different senses were was associated with acts of literacy, instead of having to manually construct the groups by reading all the sentences.
 

	
