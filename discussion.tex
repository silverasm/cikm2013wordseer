




\section{Discussion and Future Work}

These case studies have illustrated the use of the agile text manipulation and visualization functionality of WordSeer to help scholars make new discoveries about the world and create new understanding from text collections; these are discoveries that most likely they would not have been able to make with any other existing tool.

Prior to using WordSeer, C.F. had general ideas about the points he wanted to make based on years of reading evidence in the field, but no way to quantify or codify those ideas based on data.  A few sessions using the tool allowed him to find changes in use in language that reflected historians' understanding of the changing attitudes between countries' relationships (verifying the decline in discussion of Japan as an economic threat to the US), but with more precision (when did China start being referred to as democratic) and with more nuance (the simplicity of the "China card" disappearing as the relationship became acknowledged as more nuanced).

\strong{Do we have time estimates for how long it took them to find this stuff, once the tool was working, or was there too much development interspersed?  Maybe we should estimate time taken though?}

Prior to using WordSeer, R.G. had spent hours reading and thinking about the literacy essays, but had not looked at them in terms of word frequency or syntactic structure.  Using WordSeer, he was able to see the documents in a new light and form new  understanding of the general trends in his students' process of learning to write.   He was also able to formulate and find evidence for a hypothesis about an increase in proficiency with advanced writing structures as they moved to successively advanced educational levels.   He used a complex combination of searching for grammatical structures, side-by-side comparisons of sets of sentences according to adjectives and their use in context, selecting a subset of words and comparing their frequency across a metadata classification.  

A key component of the use of the tool in these examples is the freedom it affords the user to pivot on words, on words' relations to other words, to create groups of words and cross them with arbitrary metadata categories, and to be able to immediately view  the context of their original sentences and documents, thus allowing "close reading" as part of the text analysis process.

As an exploratory data analysis system, WordSeer aids in the formulation of hypotheses, and the accumulation of evidence in favor or in refutation of hypothesis.  It is important when doing EDA to externally verify any hypotheses formed, preferably with evidence external to the text.  For instance, Grimmer \cite{grimmer} describes methods to validate hypotheses formed from politician's press releases, such as comparing categories formed from these documents to candidate's voting records.  Assessments of literacy essays can be compared to learning outcomes and grades, for example.  

It is also important to note that the analyst must keep oneself honest, and look for evidence to counteract ones claims.  For instance, C.F. must be sure to seek examples of word usage that counters his views of US opinions of China and Japan during the aforementioned time period.  A significant improvement to the tool would be a way to help users try to disprove any hypothesis that they think the tool has helped them to find.


Areas of Improvement needed:
	Performance Improvements
	Medium sized text collections, not huge ones
	Window management
	Better history
	More control over defaults
	Clarify counts in sentences vs documents
	
Requested additional features:
	Sentence structure recognition
		Chris' comparatives
		Rex's sentence structure finder?
	Incorporate a learning model based on marked up sentences
	
