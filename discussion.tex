




\section{Discussion and Future Work}

These case studies have illustrated the use of the agile text manipulation and visualization functionality of WordSeer to help scholars make new discoveries about the world and create new understanding from text collections; these are discoveries that most likely they would not have been able to make with any other tool available.



Prior to using WordSeer, R.G. had spent hours reading and thinking about the literacy essays, but had not looked at them in terms of word frequency or syntactic structure.  Using WordSeer, he was able to see the documents in a new light and form new  understanding of the general trends in his students' process of learning to write.   He was also able to formulate and find evidence for a hypothesis about an increase in proficiency with advanced writing structures as they moved to successively advanced educational levels.   He used a complex combination of searching for grammatical structures, side-by-side comparisons of sets of sentences according to adjectives and their use in context, selecting a subset of words and comparing their frequency across a metadata classification.

A key component of the use of the tool in these examples is the freedom it affords the user in being able to pivot on words, on word's relations to other words, on creating groups of words and crossing them with arbitrary sets of metadata categories, and on being able to immediately view the use of words in the context of their original sentences and documents.

As an exploratory data analysis system, WordSeer
This is about creating hypotheses, finding evidence for them, and then checking them with other means.


Areas of Improvement needed:
	Performance Improvements
	Medium sized text collections, not huge ones
	Window management
	Better history
	More control over defaults
	Clarify counts in sentences vs documents
	
Requested additional features:
	Sentence structure recognition
		Chris' comparatives
		Rex's sentence structure finder?
	Incorporate a learning model based on marked up sentences
	
